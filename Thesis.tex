%---------------------------------------------------------------------------%
%-                                                                         -%
%-                           LaTeX Template                                -%
%-                                                                         -%
%---------------------------------------------------------------------------%
%- Copyright (C) Huangrui Mo <huangrui.mo@gmail.com> 
%- This is free software: you can redistribute it and/or modify it
%- under the terms of the GNU General Public License as published by
%- the Free Software Foundation, either version 3 of the License, or
%- (at your option) any later version.
%---------------------------------------------------------------------------%
%->> Document class declaration
%---------------------------------------------------------------------------%
\documentclass[twoside]{Style/ucasthesis}%
%- Multiple optional arguments:
%- [<oneside|twoside|print>]% oneside eprint, twoside eprint, or paper print
%- [fontset=<adobe|none|...>]% specify font set instead of automatic detection
%- [scheme=plain]% thesis writing of international students
%- [draftversion]% show draft version information
%- [standard options for ctex book class: draft|paper size|font size|...]%
%---------------------------------------------------------------------------%
%->> Document settings
%---------------------------------------------------------------------------%
\usepackage[authoryear,list]{Style/artratex}% document settings
%- usage: \usepackage[option1,option2,...,optionN]{artratex}
%- Multiple optional arguments:
%- [bibtex|biber]% set bibliography processor and package
%- [<numbers|super|authoryear|alpha>]% set citation and reference style
%- <numbers>: textual: Jones [1]; parenthetical: [1]
%- <super>: textual: Jones superscript [1]; parenthetical: superscript [1]
%- <authoryear>: textual: Jones (1995); parenthetical: (Jones, 1995)
%- <alpha>: textual: not available; parenthetical: [Jon95]
%- [geometry]% reconfigure page layout via geometry package
%- [lscape]% provide landscape layout environment
%- [xhf]% disable header and footer via fancyhdr package
%- [color]% provide color support via xcolor package
%- [background]% enable page background
%- [tikz]% provide complex diagrams via tikz package
%- [table]% provide complex tables via ctable package
%- [list]% provide enhanced list environments for algorithm and coding
%- [math]% enable some extra math packages
%- [xlink]% disable link colors
\usepackage{Style/artracom}% user defined commands
%---------------------------------------------------------------------------%
%->> Document inclusion
%---------------------------------------------------------------------------%
%\includeonly{Tex/Chap_1,...,Tex/Chap_N}% selected files compilation
%---------------------------------------------------------------------------%
%->> Document content
%---------------------------------------------------------------------------%
%-
%-> Titlepage information
%-
%---------------------------------------------------------------------------%
%->> Titlepage information
%---------------------------------------------------------------------------%
%-
%-> 中文封面信息
%-
\confidential{}% 密级:只有涉密论文才填写
\schoollogo[scale=0.095]{ucas_logo}% 校徽
\title{三角形计数近似算法调研}% 论文中文题目
\author{刘丁玮}% 论文作者
% \advisor{刘青泉~研究员~中国科学院力学研究所\\}% 指导教师:姓名 专业技术职务 工作单位
%\advisor{指导教师一\\指导教师二\\指导教师三}% 多行指导教师示例
\degree{202218013229010}% 学位:学士、硕士、博士
% \degreetype{理学}% 学位类别:理学、工学、工程、医学等
\major{}% 二级学科专业名称
\institute{}% 院系名称
%\institute{中国科学院力学研究所\\流固耦合实验室}% 多行院系名称示例
\date{2022~年~6~月}% 毕业日期:夏季为6月、冬季为12月
%-
%-> 英文封面信息
%-
\AUTHOR{Mo Huangrui}% 论文作者
\ADVISOR{Supervisor: Professor Liu Qingquan}% 指导教师
\DEGREE{Master}% 学位:Bachelor, Master, Doctor, Postdoctor。封面据英文学位名称自动切换,需确保拼写准确
\DEGREETYPE{Natural Science}% 学位类别:Philosophy, Natural Science, Engineering, Economics, Agriculture 等
\MAJOR{Fluid Mechanics}% 二级学科专业名称
\INSTITUTE{Institute of Mechanics, Chinese Academy of Sciences}% 院系名称
\DATE{June, 2014}% 毕业日期:夏季为June、冬季为December
%---------------------------------------------------------------------------%
%
\begin{document}
%-
%-> Frontmatter: title page, abstract, content list, symbol list, preface
%-
\frontmatter% initialize the environment
\maketitle

\intobmk\chapter*{摘\quad 要}% 显示在书签但不显示在目录
\setcounter{page}{1}% 开始页码
\pagenumbering{Roman}% 页码符号

计算和枚举图数据中的拓扑结构是大规模图数据分析中的重要任务。
三角形计数是一个典型的任务,它常被用来计算图中的节点传递性(vertex-transitivity),以了解图的演变。
在现实生活中,三角形计数常用于社区发现、链接预测、垃圾邮件过滤等等,近年来得到了数据挖掘界的广泛关注。
虽然三角形计数任务简单,但是大多数现有的算法不能很好地扩展到具有数百万甚至数十亿级别节点的大规模图数据上。
为了规避这一限制,近年来的研究工作提出了近似的三角形计数以及在分布式集群上运行的三角形计数方法。

% TODO:待修改
在本文中,我们讨论了现有的三角形计数方法,从顺序到并行,从单机到分布式,从精确到近似,从离线到流式。
我们还介绍了在一个统一的实现框架下建立的一组近似三角形计数方法的性能比较实验结果。
最后,我们对这一方向的未来工作进行了讨论。

\keywords{三角形计数、近似算法}% 中文关键词
% \input{Tex/Frontmatter}% title page, abstract
{% content list region
\linespread{1.2}% local line space
\intobmk*{\cleardoublepage}{\contentsname}% add link to bookmark
\tableofcontents% content catalog
% \intobmk*{\cleardoublepage}{\listfigurename}% add link to bookmark
% \listoffigures% figure catalog
% \intobmk*{\cleardoublepage}{\listtablename}% add link to bookmark
% \listoftables% table catalog
}
% \input{Tex/Prematter}% symbol list, preface content
%-
%-> Mainmatter
%-
\mainmatter% initialize the environment
\section{摘要}

计算和枚举图数据中的拓扑结构是大规模图数据分析中的重要任务。
三角形计数是一个典型的任务,它常被用来计算图中的节点传递性(vertex-transitivity),以了解图的演变。
在现实生活中,三角形计数常用于社区发现、链接预测、垃圾邮件过滤等等,近年来得到了数据挖掘界的广泛关注。
虽然三角形计数任务简单,但是大多数现有的算法不能很好地扩展到具有数百万甚至数十亿级别节点的大规模图数据上。
为了规避这一限制,近年来的研究工作提出了近似的三角形计数以及在分布式集群上运行的三角形计数方法。

% TODO:待修改
在本文中,我们讨论了现有的三角形计数方法,从顺序到并行,从单机到分布式,从精确到近似,从离线到流式。
我们还介绍了在一个统一的实现框架下建立的一组近似三角形计数方法的性能比较实验结果。
最后,我们对这一方向的未来工作进行了讨论。

\section{引言}

图结构数据广泛存在于众多领域中,包括社交网络、通信网络和生物网络等待。
尽管这些领域的图数据在结构组成上存在差异,但一些拓扑结构,特别是三角形,在所有不同领域的图中大量出现。
现实世界图中三角形的丰富性促使研究者发明了一些指标,如集聚系数(clustering coefficient)\citep{watts1998collective}、传递率(transitivity ratio)\citep{holland1971transitivity}等来描述和分析图。
社交网络中的三角形也被广泛研究,并从各种社会科学理论中得到解释,如同质性(homophily)\citep{mcpherson2001birds}和传递性(transitivity)\citep{holland1971transitivity}。
所有这些研究的一个关键计算任务是计算网络中三角形的数量。

三角形计数在现实生活中有很多应用。
其中最著名的是计算一个图的传递率,即为一个图中三角形和三链(长度为2的路径)数量的比率。
鉴于三链的数量可以简单地从节点度数中计算出来,因此传递率计算等价于三链计数任务。
集聚系数是用来描述一个图中节点之间结集成团的程度的系数。具体来说,是一个点的邻接点之间相互连接的程度。
例如生活社交网络中,你的朋友之间相互认识的程度。
集聚系数和传递率都被用作图分析和图演化模型的关键指标\citep{aggarwal2014evolutionary}。
\cite{becchetti2008efficient}利用本地三角形的分布来检测网络垃圾邮件。
垃圾邮件主机的局部三角形频率分布与非垃圾邮件主机的三角形频率分布有明显的不同。
\cite{eckmann2002curvature}的工作使用三角形的分布来揭示万维网中隐藏的主题结构。
万维网图中三角形密集的连接区域代表了一个共同的主题
\cite{bar2002reductions}将三角形计数用于数据库的查询计划优化。
社区发现中也常使用重叠三角形($k$-cliques)\citep{palla2005uncovering}。

虽然三角形计数在算法上似乎是一个简单的任务,
早期的研究主要关注渐进时间复杂度(asymptotic computational complexity)\citep{itai1978finding, alon1997finding}。
但在大规模图数据中做计算时,其高计算复杂度对计算效率提出了很大挑战。
为了提升计算效率,在最近的许多工作中,通过抽样进行近似三角形计数是一个非常热门的方向\citep{tsourakakis2008fast,rahman2014sampling,rahman2013approximate,tsourakakis2009doulion,jha2015space,seshadhri2013triadic}。
研究人员试图通过在多核并行计算或分布式环境中运行的算法来提高效率\citep{rahman2013approximate,tsourakakis2009doulion,suri2011counting}。
例如,三角计算算法已经被提出用于各种数据访问场景,这些场景与传统的内存访问不同,包括限制性访问\citep{rahman2014sampling}和流式数据访问\citep{seshadhri2013triadic,buriol2006counting}。

三角形计数算法的计算复杂度是衡量其效率的一个很好的指标。
但在现实生活中,即使两种算法的计算复杂度相同,其执行时间也会有很大差异。
这一事实的主要原因是计算复杂性的隐藏常数,它取决于输入图的各种属性,例如图的稀疏性。
现实生活中的大规模图数据是非常稀疏的,其中边的数量通常是节点数量与一个常数的积;
换句话说,一个节点的平均度是常数。
另一个重要的特性是,现实世界图的度分布是倾斜的。
虽然图的平均度数是恒定的,但总有几个节点拥有非常大的度数。
这种现象通常被称为幂律(power-law)分布\citep{barabasi1999emergence},这对三角形计数算法的性能有很大影响。

\section{组织}

\section{精确三角形计数}

\section{近似三角形计数}

\section{分布式并行三角形计数}

\section{实验}

\section{总结}% main content
%-
%-> Appendix
%-
\cleardoublepage%
% \appendix% initialize the environment
% \input{Tex/Appendix}% appendix content
%-
%-> Backmatter: bibliography, glossary, index
%-
\backmatter% initialize the environment
\intotoc*{\cleardoublepage}{\bibname}% add link to toc
\artxifstreq{\artxbib}{bibtex}{% enable bibtex
    \bibliography{Biblio/ref}% bibliography
}{%
    \printbibliography% bibliography
}
\end{document}
%---------------------------------------------------------------------------%

