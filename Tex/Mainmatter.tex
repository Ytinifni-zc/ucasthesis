
% \chapter{摘要}

% 计算和枚举图数据中的拓扑结构是大规模图数据分析中的重要任务。
% 三角形计数是一个典型的任务,它常被用来计算图中的节点传递性(vertex-transitivity),以了解图的演变。
% 在现实生活中,三角形计数常用于社区发现、链接预测、垃圾邮件过滤等等,近年来得到了数据挖掘界的广泛关注。
% 虽然三角形计数任务简单,但是大多数现有的算法不能很好地扩展到具有数百万甚至数十亿级别节点的大规模图数据上。
% 为了规避这一限制,近年来的研究工作提出了近似的三角形计数以及在分布式集群上运行的三角形计数方法。

% % TODO:待修改
% 在本文中,我们讨论了现有的三角形计数方法,从顺序到并行,从单机到分布式,从精确到近似,从离线到流式。
% 我们还介绍了在一个统一的实现框架下建立的一组近似三角形计数方法的性能比较实验结果。
% 最后,我们对这一方向的未来工作进行了讨论。

\chapter{引言}

图结构数据广泛存在于众多领域中,包括社交网络、通信网络和生物网络等待。
尽管这些领域的图数据在结构组成上存在差异,但一些拓扑结构,特别是三角形,在所有不同领域的图中大量出现。
现实世界图中三角形的丰富性促使研究者发明了一些指标,如集聚系数(clustering coefficient)\citep{watts1998collective}、传递率(transitivity ratio)\citep{holland1971transitivity}等来描述和分析图。
社交网络中的三角形也被广泛研究,并从各种社会科学理论中得到解释,如同质性(homophily)\citep{mcpherson2001birds}和传递性(transitivity)\citep{holland1971transitivity}。
所有这些研究的一个关键计算任务是计算网络中三角形的数量。

三角形计数在现实生活中有很多应用。
其中最著名的是计算一个图的传递率,即为一个图中三角形和三链(长度为2的路径)数量的比率。
鉴于三链的数量可以简单地从节点度数中计算出来,因此传递率计算等价于三链计数任务。
集聚系数是用来描述一个图中节点之间结集成团的程度的系数。具体来说,是一个点的邻接点之间相互连接的程度。
例如生活社交网络中,你的朋友之间相互认识的程度。
集聚系数和传递率都被用作图分析和图演化模型的关键指标\citep{aggarwal2014evolutionary}。
\cite{becchetti2008efficient}利用本地三角形的分布来检测网络垃圾邮件。
垃圾邮件主机的局部三角形频率分布与非垃圾邮件主机的三角形频率分布有明显的不同。
\cite{eckmann2002curvature}的工作使用三角形的分布来揭示万维网中隐藏的主题结构。
万维网图中三角形密集的连接区域代表了一个共同的主题。
\cite{bar2002reductions}将三角形计数用于数据库的查询计划优化。
社区发现中也常使用重叠三角形($k$-cliques)\citep{palla2005uncovering}。

虽然三角形计数在算法上似乎是一个简单的任务,
早期的研究主要关注渐进时间复杂度(asymptotic computational complexity)\citep{itai1978finding, alon1997finding}。
但在大规模图数据中做计算时,其高计算复杂度对计算效率提出了很大挑战。
为了提升计算效率,在最近的许多工作中,通过采样进行近似三角形计数是一个非常热门的方向\citep{tsourakakis2008fast,rahman2014sampling,rahman2013approximate,tsourakakis2009doulion,jha2015space,seshadhri2013triadic}。
% 研究人员试图通过在多核并行计算或分布式环境中运行的算法来提高效率\citep{rahman2013approximate,tsourakakis2009doulion,suri2011counting}。
% 例如,三角计算算法已经被提出用于各种数据访问场景,这些场景与传统的内存访问不同,包括限制性访问\citep{rahman2014sampling}和流式数据访问\citep{seshadhri2013triadic,buriol2006counting}。

三角形计数算法的计算复杂度是衡量其效率的一个很好的指标。
但在现实生活中,即使两种算法的计算复杂度相同,其执行时间也会有很大差异。
这一事实的主要原因是计算复杂性的隐藏常数,它取决于输入图的各种属性,例如图的稀疏性。
现实生活中的大规模图数据是非常稀疏的,其中边的数量通常是节点数量与一个常数的积;
换句话说,一个节点的平均度是常数。
另一个重要的特性是,现实世界图的度分布是倾斜的。
虽然图的平均度数是恒定的,但总有几个节点拥有非常大的度数。
这种现象通常被称为幂律(power-law)分布\citep{barabasi1999emergence},这对三角形计数算法的性能有很大影响。

\chapter{基本定义、常用指标与基础采样方法}

给定输入图$G(V,E)$,其中$V$表示图的节点集合,$E$表示边集合,$n,m$分别表示节点数量$|V|$和边数量$|E|$。
为了方便本文调研三角形计数的研究工作,我们假定本文中的图$G$均为简单无向图,即任意两个节点$u$和$v$间如果存在邻边,则至多存在一条,记为$e=(u,v)$,其中$u < v$。
我们使用$d(u)$表示节点$u$的度数,$adj(u)$表示节点$u$的邻居节点集合,$inc(u)$表示节点$u$的邻边集合,$inc(e)$表示边$e$的邻接节点集合。
最大的节点度数记为$d_{max}$。

\section{三角形和三链}

连通三元组$<u,v,w>$对于节点$v$而言是一个以节点$v$为中心的长度为2的路径。
如果另外两个节点$u,w$也是有边相连的,那么称此三元组为三角形(闭三元组),否则称之为三链(开三元组)。
一个三角形包含三个闭三元组,因为三个节点都可以视作三元组的中心。

我们使用符号$\Pi_v$表示以节点$v$为中心的三元组的集合。
图$G=(V,E)$中的所有三元组集合记为$\Pi$,是所有节点的三元组集合的并集,$\Pi=\bigcup_{u\in V}\Pi_v$。
若每个节点的度数均已知,则所有三元组的数量可依下式计算,
\begin{equation}
    |\Pi|=\sum_{v\in V}|\Pi_v|=\sum_{v\in V}\binom{d(v)}{2} 
    \label{eq:triple_num}
\end{equation}
我们分别记开三元组与闭三元组的集合为$\Pi^\angle $和$\Pi^\vartriangle $,以节点$v$为中心的开闭三元组集合分别为$\Pi^\angle_v$和$\Pi^\vartriangle_v$。
注意,在集合$\Pi^\vartriangle $中,三角形中的任一节点都贡献一个不同的三元组。
我们使用$\Lambda$表示图$G$中不同三角形的集合。
显然,$|\Pi^\vartriangle|=3|\Lambda|$。
记图$G$中的三角形个数为$t(G)$,则
\begin{equation}
    t(G)=|\Lambda|=\frac{1}{3}|\Pi^\vartriangle|=\frac{1}{3}\sum_{v\in V}|\Pi^\vartriangle_v|
    \label{eq:triangle_num}
\end{equation}

当且仅当图为全连接时,图中三角形数量最多,为$\left(\begin{array}{l}n \\3\end{array}\right)$。因此任意图有$t(G)=O(n^{3})$。
从边的角度考虑,由于$m\le n^{2}$,$t(G)=O(m^{\frac{3}{2}})$。

\section{计数、枚举与采样三角形}

对于一个给定的图$G$,三角形计数(counting)的任务是获得公式\ref{eq:triangle_num}中定义的数字$t(G)$。
另一方面,三角形枚举(enumerating)任务是枚举$\Lambda$中的所有元素,即列出给定图中所有不同的三角形。
枚举是一个比计数成本更高的任务,因为前者能立即解决后者,但后者不一定能解决前者。
尽管如此,对于许多现实生活中的应用,人们可能需要枚举三角形,而不是简单地找出它的总数量,所以计数和枚举任务都有其自身的优点。
三角形的采样是为了获得$\Lambda$的一个子集,通常子集的大小是一个用户定义的参数。
根据采样算法,样本集中的三角形可以是统一选择的(每个三角形都是以统一的概率采样),也可以是以有偏见的概率采样。
有时,我们只对找到与给定节点相邻的三角形的数量感兴趣。
这项任务被称为局部三角形计数。
局部三角形计数对于确定一个给定节点的集聚系数非常重要。

\section{指标}

\textbf{集聚系数}。
集聚系数是一个表示图中节点的集聚倾向的度量。
当这个度量定义在图的节点上时,它被称为局部集聚系数。
对于一个给定的节点$u$,其局部集聚系数$C(u)$是$u$的邻居中自己是邻居的部分。
\begin{equation}
    C(u)=\frac{|(v,w):(v,w)\in E\land v,w\in adj(u)|}{adj(u)(adj(u)-1)/2}
    \label{eq:clustering_coef}
\end{equation}
节点的局部集聚系数的平均值被称为图的集聚系数。

\textbf{传递性}。
图$G$的传递性为闭三元组的数量与三元组的总数的比值,记为$\gamma(G)$,
\begin{equation}
    \gamma(G)=\frac{|\Pi^\vartriangle|}{|\Pi|}=\frac{\Pi^\vartriangle}{\Pi^\vartriangle+\Pi^\angle}
    \label{eq:transitivity}
\end{equation}
根据公式\ref{eq:triangle_num}和\ref{eq:transitivity}可得图$G$的三角形数量$t(G)$可以从其传递性计算,
\begin{equation}
    t(G)=\frac{1}{3}\cdot \gamma(G) \cdot |\Pi| 
    \label{eq:tc_trans}
\end{equation}

\section{采样}

\textbf{Metropolis–Hastings(MH)算法}。
一些近似的三角形计数方法使用基于随机游走(random walk)的间接采样策略对三角形或三角形进行采样,也被称为马尔科夫链-蒙特卡洛(Markov Chain Monte Carlo, MCMC)采样。
Metropolis-Hastings(MH)算法是MCMC算法的一个变种;
它的目标是从某个分布$\pi(x)$中抽取样本,称为目标分布,其中$\pi(x)=f(x)/K$,$f(x)$是任何为群体对象$x$分配非负实值的函数,表示其在采样方面的可取性;
$K$是一个归一化常数,使对象上的$\pi(x)$之和等于1。
通常情况下,$K$是未知的或难以计算的。
MH算法与随机游走一起用于执行MCMC采样,为此,MH算法从目标分布中抽取一连串的样本,如下所示
\begin{enumerate}
    \item 选择初始化状态($x$)满足$f(x)>0$。
    \item 从当前状态$x$开始,它使用分布$q(x, y)$对一个状态$y$进行采样,称为提议分布(proposal distribution)。
    \item 然后,它计算接受概率$\alpha (x,y)$(公式\ref{eq:mh-sample}),并以概率$\alpha (x,y)$接受移动到$y$的提议。这个过程一直持续到马尔科夫链达到一个静止分布。
    \begin{equation}
        \alpha (x,y)=\min\Bigg(\frac{\pi(y)q(y,x)}{\pi(x)q(x,y)},1 \Bigg)=\min\Bigg(\frac{f(y)q(y,x)}{f(x)q(x,y)},1 \Bigg)
        \label{eq:mh-sample}
    \end{equation}
\end{enumerate}

\textbf{重要性采样}。
重要性采样(Importance Sampling, IS)是一种采样策略,用于估计函数$f(x)$相对于某个分布$p(x)=\tilde{p}(x)/K$(称为目标分布)的期望值,而样本实际上是从一个不同的分布$q(x)$获得的,称为提议分布。
当从分布$q$中取样比较容易,但我们需要获得相对于不同分布$p$的期望值时,IS就很有用。
例如,对于三角形计数,我们想从一个均匀分布中获得三元组样本,即目标分布$p$是均匀的,但从一个有偏差的分布中取样可能更容易,如$q$。
利用IS的思想,$f(x)$相对于目标分布的期望值等于
\begin{equation}
    \mathbb{E}_{p}[f(x)]=\sum_{i=1}^{S} f\left(x_{i}\right) \cdot w\left(x_{i}\right)
    \label{eq:IS}
\end{equation}
其中
\begin{equation}
    w\left(x_{i}\right)=\frac{\widetilde{p}\left(x_{i}\right) / q\left(x_{i}\right)}{\sum_{j=1}^{S} \tilde{p}\left(x_{j}\right) / q\left(x_{j}\right)}
\end{equation}

% \chapter{组织}

\chapter{精确三角形计数}
本章将讨论精确三角形计数算法及其复杂度分析。
在随机内存访问的一般假设下,我们认为$O(1)$时间内能够获取任意节点的邻接向量,且节点$u$的邻接向量$adj(u)$是有序的。
因此,我们能够在$O(\lg n)$时间内由二分查找确认边$(u,v)$是否存在。
% $O(\lg n)$是最差情况,仅可能出现在高度节点处。
考虑到现实世界图中度的分布,这些二分查找大部分发生在小邻接向量上,只有极少部分发生在大邻接向量,即具有高度的节点处。
二分查找的成本被均摊。我们可以假设确认边存在的开销是常量。
另外,如果采用哈希表,确认边是否存在的开销是$O(1)$。

最简单的精确三角形计数算法为,枚举图中所有不同的三节点集$\{u,v,w\}$,测试它们能否构成三角形。
显然,图中三节点集的数量为$O(n^{3})$,这种朴素算法的复杂度为$O(n^{3})$,并能枚举所有的三角形。
任何枚举图中所有三角形的算法最坏情况下的复杂度都是$O(n^{3})$或$O(m^{\frac{3}{2}})$,此时图中三角形数量最多,为$\frac{n}{3}$。

研究者们提出了许多具有更好的运行性能的精确三角形计数算法。
这些工作大致可以分为两类,计数算法(无枚举)和枚举算法。
其中一些计数算法在最坏情况下的时间复杂度远好于$O(n^{3})$。

\section{计数算法}
最早的无需枚举的三角形计数算法基于邻接矩阵乘。
如果$A$是无向图$G$的邻接矩阵,那么$A^{3}$的对角元素为对应顶点所在三角形数目的两倍。
$t(G)=\frac{1}{6}\text{Tr}(A^{3})$。 
该算法的时间复杂度为$O(n^{3})$,由矩阵乘操作的复杂度决定。
目前快速矩阵乘算法可以达到$O(n^{\omega})$的复杂度,$\omega$可以取到2.373\citep{le2014powers}。

\cite{alon1997finding}提出了计数算法AYZ,将时间复杂度降低到$O(m^{\frac{2\omega}{\omega +1}})$。
算法将节点集划分为低度顶点$V_{\text {low }}=\{v \in V: \mathrm{d}(v) \leq \beta\}$和高度顶点$V_{\text {high }}=V \backslash V_{\text {low }}$。
其中$\beta=m^{\omega-1 / \omega+1}$。
对于低度顶点,至多有$m \cdot \beta$条路径需要检测是否构成三角形,时间复杂度为$O(m \beta)$。
对于剩余的高度顶点,至多有$2m/ \beta$个,计数算法的复杂度为$O((m/ \beta)^{\omega})$。
因此算法的总时间复杂度为$O(m \beta + (m/ \beta)^{\omega} )=O( m^{\frac{2\omega}{\omega +1}})$。
如果$\omega = 3$,AYZ算法的复杂度为$O(m^{\frac{3}{2}})$。

\section{枚举算法}
基于枚举的算法能够列出图中全部三角形,计数反而成为微不足道的任务。
直接返回的三角形列表能够用于下游任务,比如社区发现\citep{palla2005uncovering}。

\cite{itai1978finding}提出了一种早期的枚举算法。算法第一步构造图$G(V,E)$的生成树$T(V,E_{T})$。
第二步检查$E_{T}$中的每条边是否满足$(\text{pred}(u),v) \in E$,即$u$的前继与$v$是否构成边。
如果满足,则找到一个三角形$(u,v,\text{pred}(u))$。
同样检查$(\text{pred}(v),u) \in E$。
这个过程将$T$中所有边所在的全部三角形进行了枚举。
第三步在$G$中删除$T$的所有边更新$G$。
算法迭代这三步直到图中没有边。
每次迭代的时间复杂度为$O(m)$,迭代最多进行$O(\sqrt{m})$次。
因此算法的时间复杂度为$O(m^{\frac{3}{2}})$。
然而这种算法需要修改图的数据结构,而这种操作开销很高。在实际运行时,该算法执行性能比其他算法差。

‘‘节点迭代’’算法检测每个节点的任意两个不同邻居是否有边。
$v$的邻居$u,w$间如果有边,则$u,v,w$构成一个三角形。
节点$v$需要检查$\left(\begin{array}{l}d(v) \\2\end{array}\right)$组节点对。
因此总时间复杂度为
$$\sum_{v\in V}\left(\begin{array}{l}d(v) \\2\end{array}\right)$$
即$O(nd^{2}_{\max})$。如果图的分布非常偏斜,比如星型结构的极端情况下,算法的时间复杂度接近$o(n^{3})$。

‘‘边迭代’’算法针对每条边检测两个端点的邻居的交集。
对于边$u,v$,如果$w$既是$u$的邻居又是$v$的邻居,那么$u,v,w$构成一个三角形。
由于交集检测的复杂度为$O(d_{max})$,该算法的时间复杂度为$O(m\cdot d_{max})$

对于无向图,上述算法都以顶点/边的视角对同一个三角形统计了三次。
避免这种冗余统计可以有效减少统计时间\citep{schank2005finding, latapy2008main}。
对于节点迭代算法,简单地按度排序节点,然后指定检测顺序(对于节点$v$,仅检测比$v$大的两个邻居是否有边)可以有效过滤冗余检测。
对于边迭代算法,类似地,对于边$(u,v)$仅统计$x \in adj(u) \bigcap adj(v)$且$x > \max \{u,v\}$条件下$u,v,x$构成的三角形。这样每个三角形仅会统计一遍。
计数算法中介绍的AYZ算法本质是结合了低度节点迭代和高度顶点矩阵乘计数,如果将高度顶点计数算法也改成枚举算法,就可以将AYZ拓展为枚举AYZ,$\beta$取$\sqrt{m}$,那么时间复杂度为$O(m^{\frac{3}{2}})$。

精确三角形计数算法最优的时间复杂度为$O(m^{\frac{3}{2}})$或$O(nd^{2}_{\max})$。
但不同算法在实际运行时,通过对冗余计算的优化等能够得到显著的性能提升。
要求精确三角形计数的任务中,在现实世界大数据集上达到可接受的运行时间成本。

\chapter{近似三角形计数}

精确三角形计数中的节点迭代法和边迭代法的时间复杂度为$O(m^{3/2})$。
这对于有十亿级别边的图数据的三角形计数而言,是非常昂贵的。
近年来,近似三角形技术备受关注。
近似三角形计数方法并不枚举三角形,而是给出一个近似的三角形计数。
另外,它们的执行时间相对较小。
对于不要求精确统计三角形数量的应用,用长运行时间换取好的近似值非常划算。
例如,为了分析图的演化模式,可以通过计算图的近似传递性来计算近似的三角形数量。

在现有研究工作中,近似三角形计数方法主要包括以下几种算法。
一、基于图稀疏化的均匀三角形采样;二、基于三元组采样;
三、基于节点或边迭代法的估计;

\section{基于图稀疏化的方法}

基于图稀疏化的三角形计数方法,通过随机删除图中的一个边子集来稀疏化图形。
然后从稀疏图中的精确三角形计数推算出原始图中的三角形计数。
由于稀疏图比原始图小得多,稀疏图中的三角形计数通常可以在很短的时间内完成,这使得近似方法大大快于精确计数方法。
由于稀疏图中的三角形是以原始图中所有三角形的均匀概率来采样的,因此这样的方法也可以被看作是基于均匀三角形采样的方法。

\cite{tsourakakis2009doulion}提出了最早的近似三角形计数的方法之一,即基于图稀疏化的方法DOULION。
给定图$G(V,E)$,对于图中任意边$e$,DOULION以$p$概率保留$e$,$1-p$概率删除$e$的方式生成稀疏图$G_s$。
然后在图$G_s$上计算精确的三角形计数得到$t(G_s)$。
如果原始图中的一个三角形的三条边都被保留至$G_s$中,则该三角形也被保留,概率为$p^3$。
因此,从图$G$到图$G_s$的三角形采样概率为$p^3$,原始图中的三角形期望数量为
$\hat{t}(G)=\frac{1}{p^3}\cdot t(G_s) $。
对百万级别节点的大规模图数据,边采样概率$p=0.01$即可保证计算数量能够近似精确的三角形数量。
$p=0.01$时,运行时间比精确计数法的运行时间几乎加快了100倍。

\cite{pagh2012colorful}提出了``有色三角形计数''算法,也是基于图稀疏化的近似三角形计数算法。
此方法为每个节点均匀分配$N$个颜色,标记为$1$至$N$。
然后保留两端节点颜色相同的边,删去其他边,得到稀疏化图$G_s$。
计算图$G_s$的精确三角形计数$t(G_s)$。
如此在图$G_s$中保留原图$G$的任意三角形的概率为$p^2,p=1/N$。
因为,如果三角形中的存在两条同色边,则另一条一定是同色边,保留两条边的概率为$p^2$。
原始图中三角形的期望数量为
$\hat{t}(G)=\frac{1}{p^2}\cdot t(G_s) $。
相比 DOULION $p^3$的采样概率,有色三角形计数$p^2$的采样概率具有更准确的三角形计数近似值。
\cite{pagh2012colorful}使用了方差分析证明了这种方法的三角形估计的近似率的概率界限。
他们还提出了这种算法的基于 MapReduce 的分布式实现方法。

\cite{etemadi2016efficient}提出了一种DOULION的变体。
与DOULION算法类似,它也对给定图$G$的边以概率$p$进行均匀采样,得到稀疏化图$G_{s}$。
不同之处在于,除统计$G_{s}$中的三角形外,该算法检测了三链中缺少的边是否在原始图中存在。
如果存在,这个三角形也将被统计。只有三角形的两条边都被采样到$G_{s}$中,三角形才会被统计。
因此,$G$中的三角形有$p^{2}$的概率在$t(G_{s})$中被统计。
那么,原始图中三角形的期望数量为$\hat{t}(G)=\frac{1}{ p^{2}} \cdot t\left(G_{\mathrm{s}}\right)$。\cite{etemadi2016efficient}分析方差提供了一种超参数$p$的取值方法。
作者还证明达到相同的准确度该算法比DOULION需要的样本数少。
该算法的准确率与\citep{pagh2012colorful}中算法相近,因为他们对三角形采用相同的采样率。

\section{基于三元组采样}
另一类方法通过采样三元组而不是三角形对图的三角形计数进行估计。
它们计算传递率的无偏估计并根据公式\ref{eq:tc_trans}估计图中三角形的总数。
传递率无偏估计由采样得到的三元组中三角形(闭三元组)的比例得到
\begin{equation}
    \hat{\gamma}=\frac{\sum_{t \in T} \mathbb{I}_{t \text { is closed }}}{|T|}
    \label{eq:app_trans}
\end{equation}
其中$\mathbb{I}$是指示变量。估计三角形数量时$|\Pi|=\sum_{i=1}^{n}\left(\begin{array}{c}
    d\left(u_{i}\right) \\
    2
    \end{array}\right)$,表示给定图中三元组总数。
    
计算传递率的无偏估计的核心是从一般的分布中采样三元组。
最简单的方法是均匀的采样节点$v$,然后从$v$的邻居中均匀的采样$u,w$,得到以$v$为中间节点的三元组$<u,v,w>$。
然而这种方法并不能均匀采样三元组,因为以$v$为中间节点的三元组有$\left|\Pi_{v}\right|=\left(\begin{array}{c}
    d(v) \\
    2
    \end{array}\right)$对其他节点而言不是均匀采样。$<u,v,w>$的采样概率为$1 /\left(n \cdot\left|\Pi_{v}\right|\right) \propto 1 /\left|\Pi_{v}\right|$。
高度节点为中心的三元组会欠采样,低度节点为中心的三元组会过采样。

\cite{schank2005approximating}提出了最早的使用三元组均匀采样近似图传递率的算法。
该算法首先采样节点$v$,采样率为以该节点为中心的三元组数量在全图三元组中所占的比例。
然后从$v$为中心的三元组中随机返回一个。
如果$\Pi_{v}$表示节点$v$为中心的三元组集合,有
$\Pi=\sum_{i=1}^{n} \Pi_{i}$,和$\left|\Pi_{v}\right|=\left(\begin{array}{c}d(v) \\2\end{array}\right)$。
该算法采样节点$v$的概率为$|\Pi_{v}|/|\Pi|$,采样三元组$<u,v,w>$的概率为
\begin{equation}
    P(\text { triple }\langle u, v, w\rangle \text { is sampled })=\frac{\left|\Pi_{v}\right|}{|\Pi|} \cdot \frac{1}{|\Pi_{v}|}=\frac{1}{|\Pi|} 
\end{equation}
利用公式\ref{eq:app_trans}可以从采样得到的三元组中估算传递率。然后依据如下公式估计图中三角形数。
\begin{equation}
    \hat{t}(G)=\frac{1}{3} \cdot \hat{\gamma} \cdot|\Pi|
\end{equation}
\cite{seshadhri2013triadic}采用了相同的方法。两个工作都使用霍夫丁不等式证明了近似误差边界。
\begin{theorem}
    (\cite{seshadhri2013triadic}) Set $k=\left\lceil 0.5 \varepsilon^{-2} \ln (2 / \delta)\right\rceil$. The algorithm $T_{d}$-wedge sampler outputs an estimate $W_{d} \cdot \bar{Y}$ for
    the $T_{d}$ with the following guarantee: $\left|W_{d} \cdot \bar{Y}-T_{d}\right|<\varepsilon W_{d}$
    with probability greater than $1 - \delta$.
\end{theorem}

\cite{al2016methods}提出了甚至在非均匀分布下采样三元组获得传递率无偏估计的方法。
该工作使用了IS的思想。如果想要无偏估计,我们需要有均匀的目标分布。
但三元组是从一个与$\frac{1}{|\Pi_{v}|}$成比例的分布中采样的。
因此使用公式\ref{eq:IS}可以得到传递率的无偏估计。
假定我们有一个三元组样本集合$T=\left\{t_{i}=\left\langle u_{i}, v_{i}, w_{i}\right\rangle\right\}_{1 \leq i \leq|T|}$,其中$v_{i}$是中间节点,那么重要性权重为
\begin{equation}
    w\left(t_{i}\right)=\frac{\left|\Pi_{v_{i}}\right|}{\sum_{j=1}^{|T|}\left|\Pi_{v_{j}}\right|}
\end{equation}
无偏估计可以简化为
\begin{equation}
    \begin{aligned}
        \hat{\gamma} &=\sum_{t_{i} \in T}\left(w\left(t_{i}\right) \cdot \mathbb{I}_{t_{i}} \text { is closed }\right) \\
        &=\frac{\sum_{t_{i} \in T}\left|\Pi_{v_{i}}\right| \cdot \mathbb{I}_{t_{i} \text { is closed }}}{\sum_{t_{j} \in T}\left|\Pi_{v_{j}}\right|}
        \end{aligned}
\end{equation}

\section{基于节点或边迭代法的估计}
\cite{rahman2013approximate}提出了一种由节点或边迭代算法拓展的简单的近似三角形计数算法。
在精确的节点迭代算法中,我们通过统计每个节点所在的三角形的精确数量计算图中三角形总数。
近似算法中,我们只需采样部分节点并统计它们所在三角形数量,然后根据采样的比例估计图中三角形总数。
类似地,基于边迭代的估计,只需要采样部分边并统计它们所在三角形数量,然后根据采样比例估计总数。
\cite{rahman2013approximate}的实验表明在巨大的真实图上,这种简单的方法同样有很好的效果。
而基于边迭代的估计方法比基于节点迭代的估计方法效果更好。
虽然这种没有均匀的采样三角形,但是由于期望是针对边的,三角形计数仍是无偏估计。

\chapter{总结}
三角形在图分析领域有着非常重要的作用。
在不同的应用领域中三角形都蕴含着丰富的信息。
在社交网络中,三角形对于理解网络的演变非常重要。
在生物网络中,一些三角形模式被发现能够表示多种生物信息。
三角形的计数和枚举都是大图数据挖掘中非常重要的任务。
早期研究者对精确算法进行优化,将时间复杂度和实际运行开销降低到可接受的范围。
仍然有很多任务并不需要精确的计数值,也不需要枚举,比如前沿的GNN模型将三角形的统计信息纳入特征。
并且精确算法在超大图上开销巨大。
近似算法在这些场景下更有竞争力。
许多研究者探索了三角形计数近似算法的可能,并在实际场景中取得了很好的结果。
因此本文从精确算法到近似算法,介绍了三角形计数的一些进展。
三角形计数的近似算法大多以采样的思路由部分估算全图的三角形数量。

