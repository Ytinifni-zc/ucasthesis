\section{摘要}

计算和枚举图数据中的拓扑结构是大规模图数据分析中的重要任务。
三角形计数是一个典型的任务,它常被用来计算图中的节点传递性(vertex-transitivity),以了解图的演变。
在现实生活中,三角形计数常用于社区发现、链接预测、垃圾邮件过滤等等,近年来得到了数据挖掘界的广泛关注。
虽然三角形计数任务简单,但是大多数现有的算法不能很好地扩展到具有数百万甚至数十亿级别节点的大规模图数据上。
为了规避这一限制,近年来的研究工作提出了近似的三角形计数以及在分布式集群上运行的三角形计数方法。

% TODO:待修改
在本文中,我们讨论了现有的三角形计数方法,从顺序到并行,从单机到分布式,从精确到近似,从离线到流式。
我们还介绍了在一个统一的实现框架下建立的一组近似三角形计数方法的性能比较实验结果。
最后,我们对这一方向的未来工作进行了讨论。

\section{引言}

图结构数据广泛存在于众多领域中,包括社交网络、通信网络和生物网络等待。
尽管这些领域的图数据在结构组成上存在差异,但一些拓扑结构,特别是三角形,在所有不同领域的图中大量出现。
现实世界图中三角形的丰富性促使研究者发明了一些指标,如集聚系数(clustering coefficient)\citep{watts1998collective}、传递率(transitivity ratio)\citep{holland1971transitivity}等来描述和分析图。
社交网络中的三角形也被广泛研究,并从各种社会科学理论中得到解释,如同质性(homophily)\citep{mcpherson2001birds}和传递性(transitivity)\citep{holland1971transitivity}。
所有这些研究的一个关键计算任务是计算网络中三角形的数量。

三角形计数在现实生活中有很多应用。
其中最著名的是计算一个图的传递率,即为一个图中三角形和三链(长度为2的路径)数量的比率。
鉴于三链的数量可以简单地从节点度数中计算出来,因此传递率计算等价于三链计数任务。
集聚系数是用来描述一个图中节点之间结集成团的程度的系数。具体来说,是一个点的邻接点之间相互连接的程度。
例如生活社交网络中,你的朋友之间相互认识的程度。
集聚系数和传递率都被用作图分析和图演化模型的关键指标\citep{aggarwal2014evolutionary}。
\cite{becchetti2008efficient}利用本地三角形的分布来检测网络垃圾邮件。
垃圾邮件主机的局部三角形频率分布与非垃圾邮件主机的三角形频率分布有明显的不同。
\cite{eckmann2002curvature}的工作使用三角形的分布来揭示万维网中隐藏的主题结构。
万维网图中三角形密集的连接区域代表了一个共同的主题
\cite{bar2002reductions}将三角形计数用于数据库的查询计划优化。
社区发现中也常使用重叠三角形($k$-cliques)\citep{palla2005uncovering}。

虽然三角形计数在算法上似乎是一个简单的任务,
早期的研究主要关注渐进时间复杂度(asymptotic computational complexity)\citep{itai1978finding, alon1997finding}。
但在大规模图数据中做计算时,其高计算复杂度对计算效率提出了很大挑战。
为了提升计算效率,在最近的许多工作中,通过抽样进行近似三角形计数是一个非常热门的方向\citep{tsourakakis2008fast,rahman2014sampling,rahman2013approximate,tsourakakis2009doulion,jha2015space,seshadhri2013triadic}。
研究人员试图通过在多核并行计算或分布式环境中运行的算法来提高效率\citep{rahman2013approximate,tsourakakis2009doulion,suri2011counting}。
例如,三角计算算法已经被提出用于各种数据访问场景,这些场景与传统的内存访问不同,包括限制性访问\citep{rahman2014sampling}和流式数据访问\citep{seshadhri2013triadic,buriol2006counting}。

三角形计数算法的计算复杂度是衡量其效率的一个很好的指标。
但在现实生活中,即使两种算法的计算复杂度相同,其执行时间也会有很大差异。
这一事实的主要原因是计算复杂性的隐藏常数,它取决于输入图的各种属性,例如图的稀疏性。
现实生活中的大规模图数据是非常稀疏的,其中边的数量通常是节点数量与一个常数的积;
换句话说,一个节点的平均度是常数。
另一个重要的特性是,现实世界图的度分布是倾斜的。
虽然图的平均度数是恒定的,但总有几个节点拥有非常大的度数。
这种现象通常被称为幂律(power-law)分布\citep{barabasi1999emergence},这对三角形计数算法的性能有很大影响。

\section{基本定义、常用指标与基础采样方法}

给定输入图$G(V,E)$,其中$V$表示图的节点集合,$E$表示边集合,$n,m$分别表示节点数量$|V|$和边数量$|E|$。
为了方便本文调研三角形计数的研究工作,我们假定本文中的图$G$均为简单无向图,即任意两个节点$u$和$v$间如果存在邻边,则至多存在一条,记为$e=(u,v)$,其中$u < v$。
我们使用$d(u)$表示节点$u$的度数,$adj(u)$表示节点$u$的邻居节点集合,$inc(u)$表示节点$u$的邻边集合,$inc(e)$表示边$e$的邻接节点集合。
最大的节点度数记为$d_{max}$。

\subsection{三角形和三链}

连通三元组$(u,v,w)$对于节点$v$而言是一个以节点$v$为中心的长度为2的路径。
如果另外两个节点$u,w$也是有边相连的,那么称此三元组为三角形(闭三元组),否则称之为三链(开三元组)。
一个三角形包含三个闭三元组,因为三个节点都可以视作三元组的中心。

我们使用符号$\Pi_v$表示以节点$v$为中心的三元组的集合。
图$G=(V,E)$中的所有三元组集合记为$\Pi$,是所有节点的三元组集合的并集,$\Pi=\bigcup_{u\in V}\Pi_v$。
若每个节点的度数均已知,则所有三元组的数量可依下式计算,
\begin{equation}
    |\Pi|=\sum_{v\in V}|\Pi_v|=\sum_{v\in V}\binom{d(v)}{2} 
    \label{eq:triple_num}
\end{equation}
我们分别记开三元组与闭三元组的集合为$\Pi^\angle $和$\Pi^\vartriangle $,以节点$v$为中心的开闭三元组集合分别为$\Pi^\angle_v$和$\Pi^\vartriangle_v$。
注意,在集合中$\Pi^\vartriangle $,三角形中的任一节点都贡献一个不同的三元组。
我们使用$\Lambda$表示图$G$中不同三角形的集合。
显然,$|\Pi^\vartriangle|=3|\Lambda|$。
记图$G$中的三角形个数为$t(G)$,则
\begin{equation}
    t(G)=|\Lambda|=\frac{1}{3}|\Pi^\vartriangle|=\frac{1}{3}\sum_{v\in V}|\Pi^\vartriangle_v|
    \label{eq:triangle_num}
\end{equation}

\subsection{计数、枚举与采样三角形}

对于一个给定的图$G$,三角形计数(counting)的任务是获得公式\ref{eq:triangle_num}中定义的数字$t(G)$。
另一方面,三角形枚举(enumerating)任务是枚举$\Lambda$中的所有元素,即列出给定图中所有不同的三角形。
枚举是一个比计数成本更高的任务,因为前者能立即解决后者,但后者不一定能解决前者。
尽管如此,对于许多现实生活中的应用,人们可能需要枚举三角形,而不是简单地找出它的总数量,所以计数和枚举任务都有其自身的优点。
三角形的抽样是为了获得$\Lambda$的一个子集,通常子集的大小是一个用户定义的参数。
根据抽样算法,样本集中的三角形可以是统一选择的(每个三角形都是以统一的概率抽样),也可以是以有偏见的概率抽样。
有时,我们只对找到与给定节点相邻的三角形的数量感兴趣。
这项任务被称为局部三角形计数。
局部三角形计数对于确定一个给定节点的集聚系数非常重要。

\subsection{指标}

\textbf{集聚系数}。
集聚系数是一个表示图中节点的集聚倾向的度量。
当这个度量定义在图的节点上时,它被称为局部集聚系数。
对于一个给定的节点$u$,其局部集聚系数$C(u)$是$u$的邻居中自己是邻居的部分。
\begin{equation}
    C(u)=\frac{|(v,w):(v,w)\in E\land v,w\in adj(u)|}{adj(u)(adj(u)-1)/2}
    \label{eq:clustering_coef}
\end{equation}
节点的局部集聚系数的平均值被称为图的集聚系数。

\textbf{传递性}。
图$G$的传递性为闭三元组的数量与三元组的总数的比值,记为$\gamma(G)$,
\begin{equation}
    \gamma(G)=\frac{|\Pi^\vartriangle|}{|\Pi|}=\frac{\Pi^\vartriangle}{\Pi^\vartriangle+\Pi^\angle}
    \label{eq:transitivity}
\end{equation}
根据公式\ref{eq:triangle_num}和\ref{eq:transitivity}可得图$G$的三角形数量$t(G)$可以从其传递性计算,
\begin{equation}
    t(G)=\frac{1}{3}\cdot \gamma(G) \cdot |\Pi| 
    \label{eq:tc_trans}
\end{equation}

\subsection{采样}

\textbf{Metropolis–Hastings(MH)算法}。
一些近似的三角形计数方法使用基于随机游走(random walk)的间接抽样策略对三角形或三角形进行抽样,也被称为马尔科夫链-蒙特卡洛(Markov Chain Monte Carlo, MCMC)抽样。
Metropolis-Hastings(MH)算法是MCMC算法的一个变种;
它的目标是从某个分布$\pi(x)$中抽取样本,称为目标分布,其中$\pi(x)=f(x)/K$,$f(x)$是任何为群体对象$x$分配非负实值的函数,表示其在抽样方面的可取性;
$K$是一个归一化常数,使对象上的$\pi(x)$之和等于1。
通常情况下,$K$是未知的或难以计算的。
MH算法与随机游走一起用于执行MCMC采样,为此,MH算法从目标分布中抽取一连串的样本,如下所示:
\begin{enumerate}
    \item 选择初始化状态($x$)满足$f(x)>0$。
    \item 从当前状态$x$开始,它使用分布$q(x, y)$对一个状态$y$进行采样,称为提议分布(proposal distribution)。
    \item 然后,它计算接受概率$\alpha (x,y)$(公式\ref{eq:mh-sample}),并以概率$\alpha (x,y)$接受移动到$y$的提议。这个过程一直持续到马尔科夫链达到一个静止分布。
    \begin{equation}
        \alpha (x,y)=\min\Bigg(\frac{\pi(y)q(y,x)}{\pi(x)q(x,y)},1 \Bigg)=\min\Bigg(\frac{f(y)q(y,x)}{f(x)q(x,y)},1 \Bigg)
        \label{eq:mh-sample}
    \end{equation}
\end{enumerate}

\textbf{重要性采样}。
重要性抽样(Importance Sampling, IS)是一种抽样策略,用于估计函数$f(x)$相对于某个分布$p(x)=\tilde{p}(x)/K$(称为目标分布)的期望值,而样本实际上是从一个不同的分布$q(x)$获得的,称为提议分布。
当从分布$q$中取样比较容易,但我们需要获得相对于不同分布$p$的期望值时,IS就很有用。
例如,对于三角形计数,我们想从一个均匀分布中获得三元组样本,即目标分布$p$是均匀的,但从一个有偏差的分布中取样可能更容易,如$q$。
利用IS的思想,$f(x)$相对于目标分布的期望值等于
\begin{equation}
    \mathbb{E}_{p}[f(x)]=\sum_{i=1}^{S} f\left(x_{i}\right) \cdot w\left(x_{i}\right)
    \label{eq:IS}
\end{equation}
其中
\begin{equation}
    w\left(x_{i}\right)=\frac{\widetilde{p}\left(x_{i}\right) / q\left(x_{i}\right)}{\sum_{j=1}^{S} \tilde{p}\left(x_{j}\right) / q\left(x_{j}\right)}
\end{equation}

\section{组织}

\section{精确三角形计数}

\section{近似三角形计数}

\section{分布式并行三角形计数}

\section{实验}

\section{总结}